\documentclass[a4paper, 14pt]{extarticle}
\usepackage{../generalPreamble}
\usepackage{../conspectFormat}
\usepackage{../nonFancyTOC}
\usepackage{../russianLocale}


\title{Экзаменационные вопросы по ЦОС}
\author{Корытов Павел, 6304 \\ СПбГЭТУ \enquote{ЛЭТИ}}
\date{\today}
\begin{document}
\maketitle

\tableofcontents{}

\section{Обобщенная схема ЦОС}
\lipsum[1] %TODO

\section{Типовые последовательности ЦОС}
\lipsum[1] %TODO

\section{Линейные дискретные системы. Описание во временной области}
\lipsum[1] %TODO

\section{Линейные дискретные системы. Описание в $z$-области}
\lipsum[1] %TODO

\section{Линейные дискретные системы. Описание в частотной области}
\lipsum[1] %TODO

\section{Основные характеристики ЛДС. Соотношение вход/выход. Устойчивость ЛДС}
\lipsum[1] %TODO

\section{$z$-преобразование и его свойства}
\lipsum[1] %TODO

\section{Структруы ЛДС}
\lipsum[1] %TODO

\section{Цифровые фильтры}
\lipsum[1] %TODO

\section{Синтех КИХ-фильтров методом окон}
\lipsum[1] %TODO

\section{Синтез КИХ-фильтров методом наилучшей равномерной (чебышевской) аппроксимации}
\lipsum[1] %TODO

\section{Синтез БИХ-фильтров}
\lipsum[1] %TODO

\section{Описание дискретных сигналов в $z$-области}
\lipsum[1] %TODO

\section{Описание дискретных сигналов в частотной области}
\lipsum[1] %TODO

\section{Дискретное преобразование Фурье (ДПФ)}
\lipsum[1] %TODO

\section{Методы непараметрического спектрального анализа}
\lipsum[1] %TODO

\section{Методы параметрического спектрального анализа}
\lipsum[1] %TODO

\section{Адаптивные фильтры и их применения}
\lipsum[1] %TODO


\end{document}
