\documentclass[a4paper, 14pt]{extarticle}

\usepackage{../conspectPreamble}

\begin{document}
\begin{titlepage}
    {\centering
        {\bfseries
            \includegraphics[height=8cm]{../res/logo.jpeg}\\
            Unity. Precision. Perfection.\\
            \vspace{3.5cm}
            \uppercase{Конспект лекций} \\
            по дисциплине \enquote{Безопасность жизнедеятельности}\\
        }
        \vspace{\fill}
    }
    \begin{tabular}{l l}
        \textbf{Лектор}: & Трусов Александр Олегович\\
        \textbf{Страниц}: &\pageref{LastPage}\\
        \textbf{Последнее обновление}: & \today{}\\ 
        \textbf{Автор}: & Корытов Павел, 6304\\
    \end{tabular}

    \vspace{2cm}
    {\centering
        Санкт-Петербург \\
        \the\year\\
    }
\end{titlepage}

\tableofcontents
\newpage

\section{Что я здесь делаю?}
\dfn{Опасность} --- совокупность явлений, процессов, объектов, способных в определённых условиях наносить ущерб здоровью человека непосредственно или косвенно, т.е. вызывать нежелательные последствия (события)

Виды опасности:
\begin{itemize}
    \item Реальная
    \item Потенциальная или скрытая
\end{itemize}

Влияние на человека:
\begin{itemize}
    \item Прямое
    \item Косвенное
\end{itemize}

\subsubsection*{Аксиомы БЖД}
\begin{enumerate}
    \item Любая человеческая деятельность потенциально опасна
    \item С развитием техники опасность увеличивается
\end{enumerate}

\subsubsection*{Классификация опасностей}
\begin{itemize}
    \item По природе происхождения
    \item По эффекту воздействия
    \item По вызываемым последствиям
    \item По приносимому ущербу
    \item По сфере проявления опасностей
\end{itemize}

\dfn{Опасный фактор (ОПФ)} --- воздействие на работающего, которое в ограниченное время может привести к травме или другому внезапному резкомму ухудшению здоровья

\dfn{Вредный фактор (ВПФ)} --- воздействие на работающего, которое в определённых условия в течение длительного времени ведет к заболеванию или ухудшению здоровья

\subsection{Теория риска}
Абсолютная безопасность, как правило, технически недостижима

\dfn{Риск} (степень риска, уровень риска) --- это частота реализации опасности
\begin{equation}
    R = nN,
\end{equation}
где:
\begin{itemize}
    \item $n$ --- значение неблагоприятых событий (несчастных случаев)
    \item $N$ --- общее число возможных событий (опасных случаев, число людей, подтверждающихся опасности или другой параметр, к которому приводится данное событие)
\end{itemize}
\dfn{Потенциальный риск}
\begin{equation}
    R = P(A) Pr, % chktex 35
\end{equation}
где:
\begin{itemize}
    \item $P(A)$ --- вероятность развития аварии на объекте, способной сформировать некий уровень опасного воздействия на человека
    \item $Pr$ --- вероятность гибели индивидума при данном уровне воздействия % chktex 35
\end{itemize}
\dfn{Допустимый риск} --- риск гибели людей, с которым может примирится государство
\begin{itemize}
    \item Допустимый риск $<10^-6$
    \item Пренебрежимо малый риск $<10^-8$
\end{itemize}
\screenshot{width=0.6\textwidth}{./img/S001.jpg}{Диаграмма}

\subsection{Стадии обеспечения безопасности}
\begin{enumerate}
    \item Идентификация опасностей
    \item Оценка риска
    \item Регулирование и контроль риска
\end{enumerate}

\subsubsection{Индентификация опасностей}
\begin{itemize}
    \item Выявление обстоятельств, которые могут потенциально приводить к травме или к заболеванию работника
    \item Выявление причин возникновения потенциального заболевания, связанного с выполняемой работой, продукцией или услугой
    \item Анализ сведений о ранее имеющиех место травмах, заболеваниях или проишествиях
\end{itemize}

\subsubsection{Оценка риска}
\begin{itemize}
    \item Определение количественных характеристик каждой опасности (вероятности реализации, уровня воздействия)\\
    Методы:
    \begin{itemize}
        \item Монографический
        \item Статистический
        \item Топографический
    \end{itemize}
    \item Определение возможных последствий реализации, сравнение с допустимыми приемлемыми уровнями воздействий
    \begin{itemize}
        \item В нормальных условия функционирования
        \item В случае отклонений в работе, возможных аварийный ситуаций
    \end{itemize}
\end{itemize}

\subsubsection{Регулирование и контроль риска}
Направления:
\begin{itemize}
    \item Исключение (замена) опасной работы (процедуры)
    \item Уменьшение вероятности возникновения опасной (аварийной) ситуации
    \item Уменьшение тяжести последствий реализации опасности (аварии)
\end{itemize}

Пути уменьшения риска:
\begin{itemize}
    \item Совершенствование технических средств и технологий
    \item Инженерные методы контроля (диагностики)
    \item Подготовка обслуживающего персонала
    \item Административные методы контроля
    \item Средства коллективной и индивидуальной защиты
    \item Подготовка противоаварийных служб
\end{itemize}

\subsubsection*{Законодательное обеспечение безопасности труда}
\textbf{Конституция РФ}:
\begin{itemize}
    \item \ldots{} в России охраняется труд и здоровье людей (ст.7)
\end{itemize}
Основной законодательный документ в производственных отношениях --- \textit{Трудовой Кодекс}.
\begin{itemize}
    \item Обеспечение приоритета сохранения жизни и здоровья работников
    \item Принятие и реализация законов и правовых актов РФ в области охраны труда
    \item Профилактика несчастных случаев и подвреждения здоровья работников
    \item Государственный надзор и контроль за соблюдением государственных нормативных требований охраны труда
    \item Проведение эффективной политики
\end{itemize}
\textbf{Статья 212}. Работодатель обязан обеспечить:
\begin{itemize}
    \item Безопасность работников
    \item Обучение безопасным методам работ
    \item Информирование работников об условиях труда
    \item Предоставление ораганам государственного контроля и профзоюзного контроля информации и документов
    \item Принятие мер по предотвращению несчастных случаев
\end{itemize}
\textbf{Статья 215}:
\begin{itemize}
    \item \ldots{} производственное оборудование, технологические процессы, материалы, в т.ч. иностранного производства, должны соответствовать государственным нормативным требованиям охраны труда и иметь декларацию о соответсвии и (или) сертификат соответствия
\end{itemize}

\textbf{УК РФ. Статья 143.} Нарушение правил охраны труды
\begin{enumerate}
    \item Нарушение правил техники безопасности или иных правил охраны труда, совершенное лицом, на котором лежали обязанности по соблюдению этих правил, если это по неосторожности причинение тяжелого или среднего вреда здоровью, наказывается лишение свободы на срок до 2-х лет
\end{enumerate}

Нормативная основа:
\begin{itemize}
    \item ГОСТ ССБТ --- система стандартов по безопасности труда
    \item Стандарты России ГОСТ Р --- с 1990 г.
    \begin{itemize}
        \item ГОСТ Р 22.YYY-ZZ --- серия ''Безопасности в ЧС``
        \item ГОСТ Р 32.X.YY-ZZ --- серия стандартов гражданской обороны (ГО)
    \end{itemize}
    \item ГОСТ МЭК
\end{itemize}

Органы надзора
\begin{itemize}
    \item Технический надзор (Госэнергоназдор, Госоргтехнадзор)
    \item Потребительский (санитарный) надзор
\end{itemize}

\subsubsection*{Организационные мероприятия}
\begin{itemize}
    \item Обучение (анализ принципов безопасной работы, моральное воздействие)
    \item Аттестация (проверка знаний, присвоение квалификационной группы по электробезопасности)
    \item Инструктажи (вводный, текущий) --- привязка общих знаний к предстоящей конкретной деятельности
    \item Проверки (плановые, контрольные)
\end{itemize}
Правила по охране труда при эксплуатации электрооборудования --- приказ Министерства труда и социальной защиты РФ от 24.07.2013 №328н
\end{document}
