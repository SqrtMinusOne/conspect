\documentclass[a4paper, 14pt]{extarticle}

\usepackage{../conspectPreamble}

\begin{document}
\begin{titlepage}
    {\centering
        {\bfseries
            \includegraphics[height=8cm]{logo.jpeg}\\
            Unity. Precision. Perfection.\\
            \vspace{3.5cm}
            \uppercase{Конспект лекций} \\
            по дисциплине \enquote{Теория автоматов и формальных языков}\\
        }
        \vspace{\fill}
    }
    \begin{tabular}{l l}
        \textbf{Лектор}: & Вербицкая\\
        \textbf{Страниц}: &\pageref{LastPage}\\
        \textbf{Последнее обновление}: & \today{}\\ 
        \textbf{Автор}: & Корытов Павел, 6304\\
    \end{tabular}

    \vspace{2cm}
    {\centering
        Санкт-Петербург \\
        \the\year\\
    }
\end{titlepage}

\tableofcontents
\newpage
\section{Введение}
Виды языков:
\begin{itemize}
    \item Естественные --- Русский, английский
    \item Искуственные
    \begin{itemize}
        \item Эсперанто, ложбан
        \item Клингонский, эльфийский
        \item C++, Python, Java, C\#, Haskell, OCaml, Perl, Coq, Agda
    \end{itemize}
\end{itemize}

\subsection{Синтаксис и семантика}
\begin{itemize}
    \item \dfn{Синтаксис} --- правила построения программ из символов
    \item \dfn{Семантика} --- правила истолкования программ
\end{itemize}
\begin{example}{Язык арифметических выражений}
    \[ 1 \bullet (2+3) / 4 - 5 \]
    \begin{itemize}
        \item Синтаксис
        \begin{itemize}
            \item \dfn{Терм} --- последовательность цифр или любое выражение в скобках
            \item \dfn{Слагаемое}, последовательность \dfn{термов}, соединенных знаками умножения и деление
            \item \dfn{Выражение} --- последовательность \dfn{слагаемых}, соединенных знаками сложения и вычитания (перед первым слагаемым может стоят минус)  
        \end{itemize}
        \item Семантика
        \begin{itemize}
            \item Значение арифметического выражения 
            \begin{itemize}
                \item $-3.75$
                \item $-4$
                \item $ \frac{-15}{4} $
            \end{itemize}
        \end{itemize}
    \end{itemize}
\end{example}

\subsection{Что такое язык? Теория множеств}
\dfn{Язык} --- множество строк

\subsubsection*{Множества}
\dfn{Множество} --- набор уникальных элементов
\begin{itemize}
    \item $x\in X$: $x$ --- элемент множества $X$ ($x$ принадлежит $X$)
    \item $x \notin X$ --- $x$ не являеся элементом $X$
    \item Уникальность, неупорядоченность: $\{13, 42\} = \{42, 13\} = \{13, 42, 13 \}$
    \item Универсальное множество --- множество всех элементов во вселенной
\end{itemize}

\subsubsection*{Подмножества}
$A$ является подмножеством $B$ тогда и только тогда, % TODO

$2^A = \{ B | B \subseteq A \}$ --- \textit{powerset}, множество всех подмножеств $A$ 

\subsubsection*{Строки}
\textit{Строка} --- последовательность символов
\begin{itemize}
    \item \textit{Алфавит} ($\Sigma$) --- конечное множество (атомарных, неделимых символов)
    \item \textit{Цепочка} --- любая конечная последовательность символов алфавита --- предложение, слово, строка, \ldots{}\\
    $\varepsilon$ --- цепочка, не содержащая ни одного символа
    \item Конкатенация строк $\alpha$ и $\beta$ --- соединение строк
    \item Обращение $a^R$ --- цепочка, символы которой записаны в обратном порядке
    \item $n$-я степень цепочки $a^n$ --- конкатенация $n$ повторений цепочки
    \item $|a|$ --- длина строки
\end{itemize}
Пример --- арифметические выражения\\
Алфавит $\Sigma = \{0, 1, \ldots, 9, +, -, *, /, (, )\}$

\subsection{Формальный язык}
\begin{itemize}
    \item $\Sigma$ --- алфавит
        \begin{itemize}
            \item $\Sigma = \{0, 1\}$
        \end{itemize}
    \item $\Sigma^*$ --- подмножетсов, содержаещее все цепочке в алфавите $\Sigma$, включая пустую цепочку
    \item $\Sigma^+ = \Sigma^* / \{\varepsilon \} $
    \item Формальный язык в алфавите $\Sigma$ --- подмножество всех цепочек в этом алфавите
    \begin{itemize}
        \item Для любого языка $L$ (в алфавите $\Sigma$) справедливо $L\subseteq \Sigma $
    \item $L = \{0, 0, 000, \ldots \} \subset {\{0,1\}}^*$
    \item $L = \{ 0, 0101, \ldots \} \subset {\{ 0, 1 \}}^*$
    \end{itemize}
\end{itemize}

\textit{Метаязык} --- язык, на котором дано описание языка
\begin{itemize}
    \item Естественный язык
    \item Язык металингвистических формул Бэкуса (БНФ)
    \item Синтаксические диаграммы
    \item Грамматики
    \item \ldots{}
\end{itemize}

\subsubsection{БНФ --- Бэкуса-Наура форма}
\begin{itemize}
    \item \textit{Символ} --- элементарное понятие языка
    \begin{itemize}
        \item $+$ --- означает сложение в языке арифметических выражений
    \end{itemize}
    \item \textit{Метапеременная} --- сложное понятие языка
    \begin{itemize}
        \item Переменной <выражение> можно обозначить выражение
    \end{itemize}
    \item \textit{Формула}
    \begin{itemize}
        \item <определяемый символ> := <посл 1> | \ldots{} | <посл n> % chktex 26
        \item В правой части формулы --- альтернатива конкатенаций строк, составленных из символов и метапеременных
    \end{itemize}
\end{itemize}

\subsubsection{Расширенная БНФ (EBNF)}
\begin{itemize} % TODO
    \item \textit{Итерация}
    \begin{itemize}
        \item $<x> := \{ <y> \}$
    \end{itemize}
    \item \textit{Условное вхождение}
    \item \textit{Скобки для группировки} 
\end{itemize}

\textit{Пример}. Арифметические выражения % TODO Example format
\[ <expr> := [-] <factor> \{ (+ | -) <factor> \}  \]
\[ <factor> := <term> \{ (* | /) <term> \} \]
\[ <term> := <number> / '('<expr>')' \]

\subsubsection{Синтаксические диаграммы Вирта}
% TODO

\subsection{Формальная грамматика}
\begin{itemize}
    \item \dfn{Порождающая грамматика} $G$ --- это четверка $\{V_T, V_N, P, S \}$
    \begin{itemize}
        \item $V_T$ --- алфавит терминальных символов
        \item $V_N$ --- алфавит нетерминальных символов
        \begin{itemize}
            \item $ V_T \cap V_N = \emptyset $
            \item $V :+ V_T \cap V_N$
        \end{itemize}
        \item $P$ --- конечное множество правил вида $\alpha \rightarrow \beta$
        \item % TODO
    \end{itemize}
\end{itemize}

\dfn{Отношение непосредственной выводимости}
\begin{itemize}
    \item $\alpha \rightarrow \beta \in P$
    \item $\gamma, \delta \in V^*$
    \item % TODO
\end{itemize}

\dfn{Отношение выводимости} --- рефлексивно-транзитивное замыкание отношения непосредственной выводимости
\begin{itemize}
    \item $\alpha_0, \alpha_1, \ldots, \alpha_n \in V^*$
    \item $\alpha_0 \Rightarrow \alpha_1 \Rightarrow \ldots \Rightarrow \alpha_n$
    \item % TODO
\end{itemize}

\[ S \rightarrow 0 | N | -N \]
\[ N \rightarrow 1A \]
\[ A \rightarrow 0A | 1A | \varepsilon \]

\[ S \Rightarrow -N \Rightarrow -1A \Rightarrow -11A \Rightarrow^* -1101A \Rightarrow -1101 \]

\subsubsection*{Свойства отношения выводимости}
\begin{itemize}
    \item Транзитивность % TODO
    \item Рефлексивность
\end{itemize}

\subsubsection*{Левосторонний вывод}
На каждом шагу заменяем самый левый нетерминал
\begin{equation} % TODO multilne eq?
    S \rightarrow AA | s\\
    A \rightarrow a\\
    A \rightarrow AA || Bb\\
    B \rightarrow c | d
\end{equation}

\subsubsection*{Язык, порождаемый грамматикой} --- всевозможные цепочки из тервиналов, которые выводятся из стартового терминала
\[ L(G) = \{ \omega \in V_T^* | S \Rightarrow^* \omega \} \]

Грамматики $G_1$ и $G_2$ эквивалентны, если равны порождаемыми ими языки

\subsubsection*{Контекстно-свободная грамматика} --- грамматика, все правила которой имеют вид $A \rightarrow \alpha, A \in V_n, \alpha \in V^*$. В левой части находится только один терминал

Проблема такой грамматики в том, что запись естественных языков (и некоторых языков программирования) в таком виде невозможожна. Например, значение оператора $*$ в C++ зависит от контекста. В Java же синтаксис контекстно-свободный.

Тем не менее, с помощью контестно-свободной грамматики можно записать отдельные элементы языков. 

\subsubsection*{Дерево вывода}
Дерево является \textit{деревом вывода} для $G = \{ V_n, V_T, P, S \} $, если
\begin{itemize}
    \item Каждый узел помечен символов из алфавита $V$
    \item Метка корня --- $S$
    \item Листья помечены терминалами, остальные узлы --- нетерминалами
    \item Если узлы $n_0, \ldots, n_k$ --- прямые потомки узла $n$, перечисленные слева направо, с метками
\end{itemize}

\textit{Теорема}. Пусть $G = \{V_N, V_T, P, S\}$ --- КС-грамматика. Вывод $S^* \Rightarrow \alpha$, где $\alpha \in V^*$, $\alpha \ne \varepsilon$ существует дерево вывода в грамматика $G$ с результатом $n$

\end{document}
