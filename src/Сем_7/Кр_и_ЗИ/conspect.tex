\documentclass[a4paper, 14pt]{extarticle}

\usepackage{../generalPreamble}
\usepackage{../conspectFormat}

\begin{document}
\begin{titlepage}
    {\centering
        {\bfseries
            \includegraphics[height=8cm]{../res/logo.jpeg}\\
            Unity. Precision. Perfection.\\
            \vspace{3.5cm}
            \uppercase{Конспект лекций} \\
            по дисциплине \enquote{Криптография и защита информации}\\
        }
        \vspace{\fill}
    }
    \begin{tabular}{l l}
        \textbf{Лектор}: & Племянников\\
        \textbf{Страниц}: &\pageref{LastPage}\\
        \textbf{Последнее обновление}: & \today{}\\ 
        \textbf{Автор}: & Корытов Павел, 6304\\
    \end{tabular}

    \vspace{2cm}
    {\centering
        Санкт-Петербург \\
        \the\year\\
    }
\end{titlepage}

\tableofcontents
\newpage

\section{Введение}
\subsection{Криптография}
\subsubsection*{Место криптографии среди других наук}
\begin{itemize}
    \item \dfn{Криптография} занимается разработкой методов (криптографических) преобразований информации с целью ее защиты от незаконных пользователей
    \item \dfn{Криптоанализ} --- занимается оценкой сильных и слабых сторон криптографических методов, а также разработкой методов, позволяющих взламывать криптографические преобразования (шифры, например)
    \item \dfn{Криптология} --- наука, занимающаяся исследованиями криптографических преобразований. Криптология состоит из двух частей --- криптография и криптоанализ
\end{itemize}
\subsubsection*{Цели информационной безопасности}
\begin{itemize}
    \item Доступность (Availability)
    \item Целостность (Integrity)
    \item Конфиденциальность (Confidentiality)
\end{itemize}
S.c. AIC-триада

\subsubsection*{Угрозы в фокусе криптографии}
\begin{itemize}
    \item Раскрытие данных
    \item Модификация данных
    \item Имитация источника
    \item Отказ от авторства
\end{itemize}

\subsubsection*{Задачи криптографии}
\begin{itemize}
    \item Обеспечение конфиденциальности --- защита содержимого информации от лиц, не имеющих к ней доступа.
    \item Обеспечение целостности --- гарантирование невозможности несанкционированного изменения информации.
    \item Обеспечение аутентификации --- разработка и внедрение методов подтверждения подлинности сторон и самой информации.
    \item Обеспечение невозможности отказа от авторства --- предотвращение возможности отказа субъектов от некоторых совершенных ими действий.
\end{itemize}

\subsubsection*{Идентификация, аутентификация и авторизация}
\begin{itemize}
    \item Идентификация --- Определение 
    \item Аутентификация --- Проверка
    \item Авторизация --- Доступ
\end{itemize}

\textit{Стеганография} --- наука о скрытой передаче информации путём сохранения в тайне самого факта передачи 

\subsubsection*{Базовая модель передачи данных}
\screenshot{width=0.8\textwidth}{./img/S001.jpg}{Базовая модель передачи данных}

\FloatBarrier{} % TODO get optimal position
\subsection{Определения}
\subsubsection*{Определения из криптографии}
\begin{itemize}
    \item \dfn{Открытый текст (plaintext)}. Данные в читаемом формате, также называемые простым текстом (cleartext).
    \item \dfn{Зашифровка (encipher)}. Действие по преобразованию исходных данных внечитаемый формат.
    \item \dfn{Шифротекст (ciphertext)} --- данные в форме, которая выглядит случайной и нечитаемой
    \item \dfn{Расшифровка (decipher)}. Действие по преобразованию шифротекста обратно в читаемую форму.
    \item \dfn{Шифр (cipher)} --- набор математических правил (алгоритм), используемых для зашифрования и расшифрования.
    \item \dfn{Секретный ключ (secret key)} --- секретная информация, используемая при зашифровании/расшифровании сообщений
    \item \dfn{Криптосистема (сryptosystem)} --- набор криптографических преобразований или алгоритмов, предназначенных для работы в единой технологической цепочке с целью решения определенной задачи защиты информационного процесса
\end{itemize}

\subsubsection*{Определения из криптоанализа}
\begin{itemize}
    \item \dfn{Криптоаналитик} (нарушитель) --- лицо (группа лиц), целью которых является прочтение или подделка защищённых криптографическими методами текстов
    \item \dfn{Атака} --- Попытки получения какой-либо скрытой информации или скрытой подделкой истинной информации
    \item \dfn{Взлом} --- Успешная проведённая атака
\end{itemize}

\subsubsection*{Виды атак}
\begin{itemize}
    \item На основе широтекста (Ciphertext Only)
    \item На основе невыбранного открытого текста (Known Plaintext)
    \item На основе выбранного открытого текста (Chosen Plaintext). Доступна аппаратура шифровальщика
    \item На основе выбранного шифротекста (Chosen Ciphertext). Доступна аппаратура принимающей стороны
\end{itemize}

\subsection{История развития криптографии}
\subsubsection*{Основные этапы развития криптографии}
\begin{itemize}
    \item Интуитивная (до начала XVI века)
    \item Формальная (XV --- XX)
    \item Научная (30-е --- 60-е годы XX века)
    \item Компьютерная (с 70-х годов XX века)
\end{itemize}

\screenshot{width=0.8\textwidth}{./img/S002.jpg}{Базовая модель классической криптосистемы}

\subsection{Интуитивная криптография}
\subsubsection{Шифр ``Сцитала''}

Открытый текст: \texttt{ПРИМЕРШИФРАСЦИТАЛА}

\begin{tabular}{|l|l|l|l|l|l|} % chktex 44
\hline % chktex 44
\textbf{П} & \textbf{Р} & \textbf{И} & \textbf{М} & \textbf{Е} & \textbf{Р} \\ \hline % chktex 44
\textbf{Ш} & \textbf{И} & \textbf{Ф} & \textbf{Р} & \textbf{А} & \textbf{С} \\ \hline % chktex 44
\textbf{Ц} & \textbf{И} & \textbf{Т} & \textbf{А} & \textbf{Л} & \textbf{А} \\ \hline % chktex 44
\end{tabular}

Текст наматывается на жезл.

Зашифрованный текст: \texttt{ПШЦРИИИФТМРАЕАЛРСА}

Методика взлома (Аристотель): на длинный конус наматывалась лента, а затем эту ленту начинали сдвигать по конусу. Там, где буквы текста формировали слова или слоги, диаметр конуса совпадал с диаметром цилиндра.

Это пример атаки методом ``грубой силы'' (brute force) --- полный перебор ключей (секретов) шифра при известном алгоритме зашифровки. Чтобы предотвратить этот тип атаки, число возможных ключей должно быть очень большим

\subsubsection*{Шифр изгороди (rail fence)}.

Открытый текст: \texttt{ПРИМЕРШИФРАИЗГОРОДИ}

\screenshot{width=0.8\textwidth}{./img/S003.jpg}{Зашифровка шифром изгороди}

Шифротекст: \texttt{ПИЕШФАЗООИРМРИРИГРД}

В современном варианте уровней может быть произвольное количество с произвольным порядком.

В приведенном варианте шифр основан только на знании алгоритма; секрета нет.

\subsubsection{Шифр Атбаш. Моноалфавитная замена}
Основан на использовании таблиц вида:

\begin{tabular}{|l|l|l|l|l|l|l|l|l|l|l|l|l|l|l|l|l|l|l|l|l|l|l|l|l|l|} % chktex 44
\hline % chktex 44
\textbf{A} & \textbf{B} & \textbf{C} & \textbf{D} & \textbf{E} & \textbf{F} & \textbf{G} & \textbf{H} & \textbf{I} & \textbf{J} & \textbf{K} & \textbf{L} & \textbf{M} & \textbf{N} & \textbf{O} & \textbf{P} & \textbf{Q} & \textbf{R} & \textbf{S} & \textbf{T} & \textbf{U} & \textbf{V} & \textbf{W} & \textbf{X} & \textbf{Y} & \textbf{Z} \\ \hline % chktex 44
\textbf{Z} & \textbf{Y} & \textbf{X} & \textbf{W} & \textbf{V} & \textbf{U} & \textbf{T} & \textbf{S} & \textbf{R} & \textbf{Q} & \textbf{P} & \textbf{O} & \textbf{N} & \textbf{M} & \textbf{L} & \textbf{K} & \textbf{J} & \textbf{I} & \textbf{H} & \textbf{G} & \textbf{F} & \textbf{E} & \textbf{D} & \textbf{C} & \textbf{B} & \textbf{A} \\ \hline % chktex 44
\end{tabular}

\subsubsection{Шифр Цезаря}
Основан на сдвиге алфавита. У Цезаря алфавита сдвигался на 3 буквы

Еще совсем недавно (в 1980-х годах) применялся метод шифрования \dfn{ROT13}, в котором использовался сдвиг алфавита на 13 букв вместо трех. Этот шифр использовался в различных онлайновых форумах для публикации запрещенной информации.

Шифры ``Изгородь'', ``Атбаш'', ``Цезаря'' основаны на знании алгоритма. Это примеры безключевых шифров. Взлом подобных шифров не являются предметом криптоанализа

\subsection{Формальная криптография}
\subsubsection{Аддитивный шифр}
\begin{itemize}
    \item Заменим буквы алфавита числами соответствующими их порядковым номерам в алфавите $0, 1, \ldots, n-1$.
    \item Представим символы открытого текста $P_i$ и шифротекста $C_i$
    \item Выбираем в качестве ключа числа $k$
    \item Шифрование: $C_i = (P_i + k) \bmod n$ 
    \item Расшифровка: $ P_i = (C_i - k) \bmod n $
\end{itemize}

Шифр уязвим к атакам методом ``грубой силы''. Множество ключей аддитивного шифра равно числу букв алфавита. Нулевой ключ, является бесполезным (зашифрованный текст будет совпадать с исходным текстом). Требуется перебор $n-1$ возможных ключей

\subsubsection{Мультипликативный шифр}
\begin{itemize}
    \item Заменим буквы алфавита числами соответствующими их порядковым номерам в алфавите $0, 1, \ldots, n-1$.
    \item Представим символы открытого текста $P_i$ и шифротекста $C_i$
    \item Выбираем в качестве ключа число $k \in \interval[open right]{1}{n} $, $k\times k^-1 \equiv 1 \bmod n$ (существует мультипликативная версия)
    \item Шифрование символа: $C_i = (P_i \times k) \bmod n$
    \item Расшифровка символа: $P_i = (C_i \times k^-1) \bmod n$
\end{itemize}

Множество ключей мультипликативного шифра равно числу ключей аддитивного шифра, имеющих мультипликативную инверсию. Требуется перебор в худшем случае n-1 возможных ключей

\subsubsection{Афинный шифр}
\begin{itemize}
    \item Комбинация аддитивного и мультипликативного шифров
    \item Ключ состоит из двух частей: $k_1, k_2$
    \item Шифрование: $C_i = (P_i \times k_1 + k_2) \bmod n$
    \item Расшифровка: $ P_i = ((C_i - k_2) \times k_1^{-1}) \bmod n $
    \item При $k=1$ --- аддитивный шифр
    \item При $k=1$ и $k_2 = 25$ --- шифр Атбаш
    \item При $k_2=0$ --- мультипликативный шифр
\end{itemize}
Сложность атаки грубой силой --- $\varphi(n)\times n$, где $\varphi(n)$ --- функция Эйлера.

Если известна биграмма $P_{i}P_{i+1}$ и её шифр $C_{i}C_{i+1}$, можно решить систему уравнений:
\begin{equation*}
    \begin{cases}
        C_i = (P_i \times k_1 + k_2) \bmod n\\
        C_{i+1} = (P_{i+1} \times k_1 + k_2) \bmod n
    \end{cases}
\end{equation*}
Т.е. определить $ k_1 = ((C_{i+1} - C_i) \times k_1^{-1} + (P_{i+1} - P_i)) \bmod n $. В случае нескольких решений ориентироваться на связность расшифрованного текста

\subsubsection{Шифр моноалфавитной подставновки}
Символы алфафита однозначно заменяются на символы другого алфавита. Ключ --- перестановка; количество перестановок --- $ n! $

Возможна атака методом \textit{частотного анализа}:
\begin{itemize}
    \item Подсчитывается частота появления каждой буквы шифротекста
    \item Полученное распределение частот сравнивается, например, со справочной таблицей частот для символов языка открытого текста
    \item Ввыдвигаются гипотезы о соответствии букв открытого текста и шифротекста
    \item Сделанные гипотезы проверяются с помощью справочных таблиц распределения биграмм и триграмм
\end{itemize}

\subsubsection{Омофонический шифр}
Идея --- сделать частоту распределения символов максимально равномерной. Наиболее частым букв исходного алфавита ставится большее количество букв алфавита шифра
\screenshot{width=0.95\textwidth}{./img/S004.jpg}{Пример омофонического шифра для английского языка}

\subsubsection{Шифр Виженера}
Шифр многоалфавитной замены.

\screenshot{width=0.95\textwidth}{./img/S005.jpg}{Пример шифра Виженера}

\begin{itemize}
    \item Заменим буквы алфавита числами соответствующими их порядковым номерам в алфавите $0, 1, \ldots, n-1$.
    \item Представим символы открытого текста $P_i$, ключа $K_i$ и шифротекста $C_i$
    \item Сформируем \dfn{гамму} повторением ключа: $G=(K_1, \ldots, K_M) \ldots (K_1, \ldots, K_m)$
    \item Шифрование символа: $C_i = (P_i + G_i) \bmod n$
    \item Расшифровка символа: $P_i = (C_i - G_i) \bmod n$
\end{itemize}
Сложность атаки грубой силы --- $ \frac{n!}{(n-m)!} $. 

При атаке шифр рассматривается как комбинации аддитивных шифров. 
\begin{enumerate}
    \item Автокорреляционный метод. Определение длины ключевого слова. 
    \begin{itemize}
        \item Шифротекст (длиной $L$) выписывается в строку. Под ней выписываются строки, полученные сдвигом влево на $t=1,2,3,\ldots$ позиций.
        \item $ \forall t $ подсчитывается $n_t$ --- число совпадений символов, находившихся на одинаковых позициях в шифротексте и его версии со сдвигом $t$
        \item Вычисляеются автокорреляционные коэффициенты $ K_t = \frac{n_t}{L-t} $
        \item Для сдвигов, кратных периоду ключа, $K_t$ будут заметно больше и будут иметь значение, близкое к индексу совпадений используемого языка (для русского языка $\approx 0.0533$)
        \item Соответствующие сдвиги $t$ берутся в качестве оценки ключа
    \end{itemize}
    \item Статистический метод. Разделение шифротекста на части, зашифрованных одинаковым символом ключа и анализ полученных частей методами статистического анализа для поиска всех символов ключа
    \begin{itemize}
        \item Анализируются фрагменты шифротекста, зашифрованные одной и той же буквой шифра
        \item По возможности применяются методы частотного анализа
    \end{itemize}
\end{enumerate}

\subsubsection{Свойства рассмотренных шифров}
\begin{itemize}
    \item Симметричность --- отправитель и получатель обладают одинаковыми секретными ключами и одинаковыми алгоритмами для зашифрования и расшифрования
    \item Поточность --- каждый символ открытого текста преобразуется в сисмвол зашифрованного текста в зависимости не только от используемого ключа, но и от его расположения в потоке открытого текста
    \item Основаны преимущественно на операциях перестановки и замены (подстановки)
\end{itemize}

\subsubsection{Шифр двойной перестановки}
Исходый текст записывается в матрицу. Ключ --- перестановка столбцов и строк в матрице. Сложность атаки грубой силы --- $n! \times m! $.

\screenshot{width=0.9\textwidth}{./img/S006.jpg}{Пример шифра двойной перестановки}

Общее название таких шифров --- шифры маршрутных перестановок.

Для расшифровки применим частотный анализ биграмм. Предпринимаются попытки определить размер столбца; затем отсеиваются гипотезы перестановок с помощью обнаружения запретных биграмм.

Желательно знание фрагментов открытого текста.

\subsubsection{Шифр Плейфера}
Военный шифр 1854 года. Первый пример блочного шифра. 

Исходный текст разбивается на блоки --- биграммы по 2 символа. Ключом является матрица $5\times$.
\begin{itemize}
    \item Если две буквы находятся в одной строке, то они заменяются на букву справа
    \item Если в одном столбце, то аналогично для столбца
    \item Если в разных, то составляется биграмма из букв, расположенных по противоположной диагонали
\end{itemize}

\screenshot{width=0.9\textwidth}{./img/S007.jpg}{Пример шифра Плейфера}

Сложность атаки грубой силой --- $25! $. 

Преимущество шифра в том, что он скрывает частоту отдельных букв, но возможна атака на основе анализа частоты биграмм. Для расшифровки полезно знание фрагментов исходного текста, например стандартной формы обращения.

Во время войны устраивались провокации, после чего на основе данных о провокации проводилась расшифровка.

\subsubsection{Шифр Хилла}
Изобретён в 1929 году. 

\screenshot{width=0.9\textwidth}{./img/S008.jpg}{Зашифровка шифром Хилла}

\screenshot{width=0.9\textwidth}{./img/S009.jpg}{Расшифровка шифра Хилла}

Требования к матрице --- она должна быть обратима, т.е. $|M| \ne 0$. В таком случае $M^{-1}$ --- мультипликативная инверсия в $\mathbb{Z}_{26}$: $M \times M^{-1} \equiv I \bmod 26 $

Сложность атаки грубой силов --- $n^{m \times m}$. 

Этот шифр совсем не сохраняет статистику исходного текста. Атака возможна на основе знания исходного текста:
\begin{itemize}
    \item Делается предположение о размере блока (например, $m$)
    \item Добываются не менее $m$ пар блоков открытого текста и шифротекста и строится уравнение $ C = P \times K $
    \item Выполняется попытка восстановить матрицу-ключ $ K = C \times P^{-1} $
    \item В случае неудачи выбирается другой размер блока $m$
\end{itemize}

\FloatBarrier{}
\subsubsection{Комбинированный шифр ADFGVX}
Ещё один военный шифр; изобретен в 1918 году.

\screenshot{width=0.9\textwidth}{./img/S010.jpg}{Шаг 1 шифра ADFGVX --- замена}

\screenshot{width=0.9\textwidth}{./img/S011.jpg}{Шаг 2 --- перестановка}

Атаки основаны на знании фрагментов открытого текста:
\begin{itemize}
    \item На основе анализа 2-х или более сообщений с одинаковым начальным текстом
    \item На основе анализа 2-х или более сообщений с одинаковым окончанием
    \item На основе сообщений одинакового размера
\end{itemize}
Определяется перестановка, а затем --- частотным анализом --- шифрующая матрица

\end{document}
