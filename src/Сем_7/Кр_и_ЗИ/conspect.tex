\documentclass[a4paper, 14pt]{extarticle}

\usepackage{../conspectPreamble}

\begin{document}
\begin{titlepage}
    {\centering
        {\bfseries
            \includegraphics[height=8cm]{logo.jpeg}\\
            Unity. Precision. Perfection.\\
            \vspace{3.5cm}
            \uppercase{Конспект лекций} \\
            по дисциплине \enquote{Криптография и защита информации}\\
        }
        \vspace{\fill}
    }
    \begin{tabular}{l l}
        \textbf{Лектор}: & Племянников\\
        \textbf{Страниц}: &\pageref{LastPage}\\
        \textbf{Последнее обновление}: & \today{}\\ 
        \textbf{Автор}: & Корытов Павел, 6304\\
    \end{tabular}

    \vspace{2cm}
    {\centering
        Санкт-Петербург \\
        \the\year\\
    }
\end{titlepage}

\tableofcontents
\newpage

\section{Введение}
\paragraph{Цели информационной безопасности}
\begin{itemize}
    \item Доступность
    \item Целостность
    \item Конфиденциальность
\end{itemize}

\paragraph{Угрозы в фокусе криптографии}
\begin{itemize}
    \item Раскрытие данных
    \item Модификация данных
    \item Имитация источника
    \item Отказ от авторства
\end{itemize}

\paragraph{Идентификация, аутентификация и авторизация}
\begin{itemize}
    \item Идентификация --- Определение
    \item Аутентификация --- Проверка
    \item Авторизация --- Доступ
\end{itemize}

\textit{Стеганография} --- наука о скрытой передаче информации путём сохранения в тайне самого факта передачи 

\subsection{Базовая модель передачи данных}
\screenshot{width=0.8\textwidth}{./placeholder.jpg}{Базовая модель передачи данных}

\FloatBarrier{} % TODO get optimal position
\subsection{Определения из криптографии}
% TODO?

\begin{itemize}
    \item \textit{Криптоаналитик} --- лицо (группа лиц), целью которых является прочтение или подделка защищённых криптографическими методами текстов
    \item \textit{Атака} --- Попытка
    \item \textit{Взлом} --- Успешная попытка
\end{itemize}

\paragraph{Виды атак}
\begin{itemize}
    \item На основе широтекста (Ciphertext Only)
    \item На основе невыбранного открытого текста (Known Plaintext)
    \item На основе выбранного открытого текста (Chosen Plaintext). Доступна аппаратура шифровальщика
    \item На основе выбранного шифротекста (Chosen Ciphertext). Доступна аппаратура принимающей стороны
\end{itemize}

\subsection{История развития криптографии}
\subsection{Основные этапы развития криптографии}
\begin{itemize}
    \item Интуитивная (до начала XVI века)
    \item Формальная (XV --- XX)
    \item Научная (30-е --- 60-е годы XX века)
    \item Компьютерная
\end{itemize}

\screenshot{width=0.5\textwidth}{./placeholder.jpg}{Базовая модель классической криптосистемы}

\subsubsection{Классические шифры}
\paragraph{Шифр ``Сцитала''}.
% TODO

Текст наматывается на жезл

\paragraph{Шифр изгороди (rail fence)}. Все четные буквы переводятся в начало, нечетные --- в конец
% TODO

В современном варианте уровней может быть произвольное количество с произвольным порядком.

В приведенном варианте шифр основан только на знании алгоритма; секрета нет.

\paragraph{Шифр Атбаш}. Моноалфавитная замена
% TODO

\paragraph{Шифр Цезаря}
% TODO
Смещение букв алфавита

\subsubsection{Формальная криптография}
\paragraph{Аддитивный шифр}
% TODO

\paragraph{Мультипликативный шифр}
Заменим буквы алфавита числами, соответствующими их порядковыми номерами в $1,\ldots{}, n-1$
% TODO 

\paragraph{Афинный шифр}
% TODO

\paragraph{Шифр моноалфавитной подставновки}
Символы алфафита однозначно заменяются на символы другого алфавита. Ключ --- перестановка; количество перестановок --- $ n! $

% TODO

Возможна атака методом \textit{частотного анализа}:
\begin{itemize}
    \item Подсчитывается частота появления каждой буквы шифротекста
    \item Полученное распределение частот сравнивается, например, со справочной таблицей частот для символов языка открытого текста
\end{itemize}

\paragraph{Омофонический шифр}
Идея --- сделать частоту распределения символов максимально равномерной. Наиболее частым букв исходного алфавита ставится большее количество букв алфавита шифра
% TODO

\paragraph{Шифр Виженера}
Шифр многоалфавитной замены.

Придумывается секретное слово и записывается под каждым словом исходного текста
% TODO

\paragraph{Свойства рассмотренных шифров}
\begin{itemize}
    \item Симметричность --- отправитель и получатель обладают одинаковыми секретными ключами и одинаковыми алгоритмами для зашифрования и расшифрования
    \item Поточность --- каждый символ открытого текста преобразуется в сисмвол зашифрованного текста в зависимости не только от используемого ключа, но и от его расположения в потоке открытого текста
    \item Основаны преимущественно на операциях перестановки и замены (подстановки)
\end{itemize}

\paragraph{Шифр двойной перестановки}
Исходый текст записывается в матрицу. Ключ --- перестановка столбцов и строк в матрице.

Общее название таких шифров --- шифры маршрутных перестановок.
% TODO

\paragraph{Шифр Плейфера}
Первый пример блочного шифра. 

Исходный текст разбивается на блоки --- биграммы по 2 символа. 
\begin{itemize}
    \item Если две буквы находятся в одной строке, то они заменяются на букву справа
    \item Если в одном столбце, то аналогично для столбца
    \item Если в разных, то составляется биграмма из букв, расположенных по противоположной диагонали
\end{itemize}

\paragraph{Шифр Хилла}
Блоки переменной длины. 

% TODO

Выполняется матрицчное преобразование --- умножение и взятие по модулю

% TODO

Для расшифрования нужна обратная матрица

\paragraph{Комбинированный шифр ADFGVX}


\end{document}
